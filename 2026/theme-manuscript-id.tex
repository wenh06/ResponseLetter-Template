%%%%%%%%%%%%%%%%%%%%%%%%%%%%%%%%%%%%%%%%%
% Academic Paper Response Letter.
% LaTeX Template
% Version 1.0 (20/07/2023)
%
% This template refers to the toolbox tutorial (in Chinese):
% https://liam.page/2016/07/22/using-the-tcolorbox-package-to-create-a-new-theorem-environment/
% Please refer to the tutorial for your customization.
%
% Template license: Creative Commons CC BY 4.0
%
% Author: Yuhan Xie
% Contact email: yhxie0107@gmail.com
% Any revision advice is welcome!
%
%%%%%%%%%%%%%%%%%%%%%%%%%%%%%%%%%%%%%%%%%


\documentclass{article}
%% Language and font encodings
\usepackage[english]{babel}
\usepackage{times} % make the section title in Times New Roman

\usepackage{booktabs}
\usepackage{tabu}
\usepackage[T1]{fontenc}

%% Sets page size and margins
\usepackage[a4paper,top=2.5cm,bottom=2cm,left=1.7cm,right=1.7cm,marginparwidth=1.75cm]{geometry}

%% Useful packages
\usepackage{amsmath}
\usepackage{amsthm}  % 处理定理和证明
\usepackage{amssymb}    % 数学符号(包括 \mathbb)
\usepackage{mathtools}
\usepackage{graphicx}
\usepackage[colorinlistoftodos]{todonotes}
\usepackage[colorlinks=true, allcolors=blue]{hyperref}
\usepackage{indentfirst}
\usepackage{graphicx}
\usepackage{subfigure}
\usepackage{float}
\usepackage{threeparttable}
\usepackage{multirow}
\usepackage{url}

\newtheorem{lemma}{Lemma}
\newtheorem{proposition}{Proposition}
\newtheorem{corollary}{Corollary}
\newtheorem{theorem}{Theorem}
\newtheorem{example}{Example}
\newtheorem{remark}{Remark}
\newtheorem{definition}{Definition}
%\newtheorem{proof}{Proof}

% define color
\definecolor{lightred}{RGB}{216,144,144}
\definecolor{slightred}{RGB}{251,243,243}
\definecolor{slightgreen}{RGB}{243,248,243}

\usepackage[most]{tcolorbox}
\tcbuselibrary{breakable,skins}
\newtcolorbox{changebox}{
  enhanced, frame hidden, breakable, sharp corners,
  leftrule=0pt, rightrule=0pt, toprule=0pt, bottomrule=0pt,
  boxsep=2pt, top=0.5em, bottom=0.5em, left=0.5em, right=0.5em,
  before skip=0.75em, after skip=0.75em,
  colframe=gray, colback = slightgreen
}
\newtcolorbox{bigquote}{
  enhanced,
  breakable,
  colback=white,        % 背景色(可改为 gray!5 之类的)
  colframe=white,       % 边框颜色(white = 无边框)
  boxrule=0pt,
  left=3em,             % 左缩进
  right=3em,            % 右缩进
  sharp corners,
  before skip=10pt,
  after skip=10pt,
  overlay={
    % 左上角的大引号
    \node[anchor=north west, xshift=.9em,
          font=\fontsize{25}{25}\selectfont, text=gray!70]
          at (frame.north west) {``};
    % 右下角的大引号
    \node[anchor=south east, xshift=-.9em, yshift=-.7em,
          font=\fontsize{25}{25}\selectfont, text=gray!70]
          at (frame.south east) {''};
  }
}


\newcommand{\rev}[1]{{\color{magenta}{#1}}}

% define maketitle environment
\makeatletter
\def\@maketitle{%
  \newpage
  \null
  \vskip 0.5em%
  \leftline{Reply to the review report on}
  \vskip 1.5em%
  \begin{center}%
  \let \footnote \thanks
    {\Large \bfseries \@title \par}%
    \vskip 1.5em%
    {\large
      \lineskip .5em%
      \begin{tabular}[t]{c}%
        \@author
      \end{tabular}\par}%
    \vskip 1em%
    {\large \@date}%
  \end{center}%
  \par
  \vskip 1.0em}
\makeatother

\title{Here Is the Title}
\author{by \quad Author1~ and ~ Author2}
\date{\today}

\begin{document}

%
%
%\newtcbtheorem[auto counter, number within = section]{cmt}{Comment}{
%	colbacktitle = black!60!white, colframe = black!60!white,
%	colback = black!5!white,
%	fonttitle=\bfseries,%fontupper=\itshape,
%}{t}

\newtcbtheorem[auto counter]{cmt}{Comment}{
	colbacktitle = black!60!white, colframe = black!60!white,
	colback = black!5!white,
	fonttitle=\bfseries,%fontupper=\itshape,
}{t}


%\newtcbtheorem[auto counter, number within = section]{cmt}{Comment}{
%	colbacktitle = black!60!white, colframe = gray,,
%	colback = {slightred},
%	fonttitle=\bfseries,%fontupper=\itshape,
%}{t}


\maketitle


We thank the editor and the reviewers for the effort in handling and reviewing our manuscript (Paper ID), "(Paper Title)". This paper [state contribution].


In the following, we address your concerns point by point. The changes have also been marked in the revised manuscript, highlighted in \rev{magenta}.

\vskip 16pt \hrule height 0.6pt

\section*{Response to Reviewer \#1}
\begin{cmt}{}{}
comment 1
\end{cmt}

\vspace{0.1cm}
\noindent
\underline{\textbf{Response:}}
\vspace{0.2cm}

Thank you for your insightful comment regarding

\cite{Ma_2024_Segment}

\vspace{0.1cm}
\noindent
\underline{\textbf{Changes:}}
\vspace{0.2cm}

\begin{changebox}
Page x: \rev{to write

\cite{Wen_2025_VLSIJ}
}
\end{changebox}


% more comments and replies, changes


\medskip

\medskip

We believe these additions and modifications address your concerns and enrich the manuscript by providing a more comprehensive understanding of our proposed algorithm's capabilities and limitations.

Thank you once again for your valuable feedback. We hope that our revised manuscript meets your approval.



%%%%%%%%%%%%%%%%%%%%%%%%%%%%% Rewiewer 1
%\newpage
%\section{Response to reviewer 1}
%
%%%%%%%%%%%%%% Comment 1.1
%\begin{cmt}{}{}
%
%Comments 1.1
%
%\end{cmt}
%\vspace{0.1cm}
%\noindent
%\underline{\textbf{Response:}}
%\vspace{0.2cm}
%
%Your Response 1.1
%
%\vspace{0.3cm}
%
%
%%%%%%%%%%%%%% Comment 1.0
%\begin{cmt*}{}{}        %% use * when do not need enumerate
%
%Comments
%
%\end{cmt*}
%\vspace{0.1cm}
%\noindent
%\underline{\textbf{Response:}}
%\vspace{0.2cm}
%
%Your Response
%
%\vspace{0.3cm}
%
%
%%%%%%%%%%%%%% Comment 1.2
%\begin{cmt}{}{}
%
%Comments 1.2
%
%\end{cmt}
%\vspace{0.1cm}
%\noindent
%\underline{\textbf{Response:}}
%\vspace{0.2cm}
%
%Your Response 1.2
%
%\vspace{0.3cm}
%
%
%%%%%%%%%%%%%% Comment 1.3
%\begin{cmt}{}{}
%
%Comments 1.3
%
%\end{cmt}
%\vspace{0.1cm}
%\noindent
%\underline{\textbf{Response:}}
%\vspace{0.2cm}
%
%Your Response 1.3
%
%\vspace{0.3cm}


\bibliographystyle{elsarticle-num}
\bibliography{
references/references-settings,
references/references-med,
references/references-ml
}


\end{document}
