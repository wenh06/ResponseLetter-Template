\documentclass[acmsmall]{acmart}
%\documentclass[manuscript,screen,review]{acmart}
%\documentclass{letter}

\usepackage{amsfonts}
\usepackage{algorithm}
\usepackage{algorithmic}
\usepackage{enumerate}
\usepackage{subfigure}
\usepackage{amsmath}
\usepackage{courier}
\usepackage{setspace}
\usepackage{multirow}
\usepackage{booktabs}
\usepackage{soul}
\usepackage{threeparttable}
\usepackage[capitalize,noabbrev]{cleveref}
\usepackage{relsize}
\usepackage{siunitx}
\DeclareSIUnit\angstrom{\text{Å}}
% \usepackage{lastpage}

\usepackage{tcolorbox}
\tcbuselibrary{skins, breakable}

% 定义环境
\newtcolorbox{bigquote}{
  enhanced,
  breakable,
  colback=white,        % 背景色(可改为 gray!5 之类的)
  colframe=white,       % 边框颜色(white = 无边框)
  boxrule=0pt,
  left=3em,             % 左缩进
  right=3em,            % 右缩进
  sharp corners,
  before skip=10pt,
  after skip=10pt,
  overlay={
    % 左上角的大引号
    \node[anchor=north west, xshift=.9em,
          font=\fontsize{25}{25}\selectfont, text=gray!70]
          at (frame.north west) {``};
    % 右下角的大引号
    \node[anchor=south east, xshift=-.9em, yshift=-.7em,
          font=\fontsize{25}{25}\selectfont, text=gray!70]
          at (frame.south east) {''};
  }
}

\theoremstyle{plain}
%\newtheorem{theorem}{Theorem}[section]
\newtheorem{theorem}{Theorem}
\newtheorem{proposition}{Proposition}
\newtheorem{lemma}{Lemma}
\newtheorem{corollary}[theorem]{Corollary}
\theoremstyle{definition}
\newtheorem{definition}[theorem]{Definition}
\newtheorem{assumption}[theorem]{Assumption}
\theoremstyle{remark}
\newtheorem{remark}{Remark}

\newcommand{\atn}[1]{\textcolor{blue}{#1}}
\newcommand{\notice}[1]{{\textrm{\textcolor{black}{#1}}}}
\newcommand{\reminder}[1]{{\textsf{\textcolor{blue}{[#1]}}}}

\newif\ifHandout
\Handoutfalse
% \Handouttrue


\ifHandout
\newcommand{\replyIncomplete}{}
\newcommand{\replyComplete}{}
\else
\newcommand{\replyIncomplete}{{\color{red}(Rely Incomplete)}}
\newcommand{\replyComplete}{{\color{green}(Rely Complete)}}
\fi

\newif\ifHighLightChanges
\HighLightChangestrue
% \HighLightChangesfalse

\ifHighLightChanges
\newcommand{\rev}[1]{{\color{blue}{#1}}}
\else
\newcommand{\rev}[1]{{#1}}
\fi

\makeatletter
\AtBeginDocument{
  \fancypagestyle{standardpagestyle}{
    \fancyhf{}
    \fancyfoot[C]{\@headfootfont Page \thepage\ of \pageref*{TotPages}}
    \renewcommand{\headrulewidth}{0pt}
    \renewcommand{\footrulewidth}{0pt}
  }
  \pagestyle{standardpagestyle}
}
\makeatother

%%
%% \BibTeX command to typeset BibTeX logo in the docs
% \AtBeginDocument{%
% 	\providecommand\BibTeX{{%
% 			\normalfont B\kern-0.5em{\scshape i\kern-0.25em b}\kern-0.8em\TeX}}}

\begin{document}
\thispagestyle{standardpagestyle}

\textbf{\larger[1] Answers to Reviewers’ Questions for [paper ID]}

\vspace{1em}

We would like to thank all reviewers and the associate editor for their helpful comments. We have addressed all the review comments below using the paragraphs starting with “\textbf{Re:}”.
All modifications in the original manuscript are marked in \textcolor{blue}{BLUE}.

The modifications are summarized as follows.

% 1) The experiments in Section 5.2 for MC-OER have been modified. To better show the improvements relative to the method in the conference paper, we have added a comparison between the improved method and the original method (the method in the conference paper) in Table 3. Moreover, we reset the time limit for the MC-OER experiment from 24 hours to 168 hours to show more wirelength results of the direct implementation of the MC-MCF model. The results are listed in Table 3 and Table 4, and better evident the advantage of proposed method.

% 2) In Section 4, we have added a subsection (Section 4.2) to demonstrate the details of using the MC-MCF Model to solve sub-problems, such as the order of solving these sub-problems.

% 3) In Section 4, Fig. 17- Fig. 20 have been modified. We have increased the color contrast between the index and node.

% 4)	The possible relation of $C_b$ and $C_d$ is modified and presented in Section 2.1. Then, we explain all the possible relation between them in details in Section 3.2.4.


% 5) The whole manuscript is read through carefully. We have tried our best to remove grammar and writing issues.

% 6) Besides, 4 references about unordered escape routing and staggered pin array have been added in the revised paper.

\vspace{2em}

\textbf{Reviewer: 1}

Comments:
% This paper is an extension of the paper “MC-MCF: A Multi-Capacity Model for Ordered Escape Routing”, which was accepted by ISQED’23.
% The original paper utilized MMCF graph to model the multi-capacity OER problem of GPA, and named its method MC-MCF.
% The original paper applied ILP to derive the routing solutions, and proposed a routing-resource-driven approach to partition the entire MC-MCF model into sub-models to boost the efficiency.
% In this extension, the authors extend their method to address the multi-capacity OER problem of both GPA and SPA.
% The efficiency of the MC-MCF method is also improved by utilizing several newly proposed techniques.
% However, there are still some issues that the authors are suggested to further improve:

\textbf{Re:}
Thank you for the positive comments.

\begin{bigquote}
\rev{Some quote.}
\end{bigquote}

\vspace{1em}

1. There are currently a few grammatical errors, for instance “… that the region include …”, please fix them.

\textbf{Re:}
Thanks a lot for the comment. We have fixed the grammatical error (see page 16).
Besides, the whole manuscript is read through carefully.
We have tried our best to remove grammar and writing issues.

\vspace{1em}

2. The quality of the figures can be further improved. For instance, the indexes of the pins in Fig. 17-Fig. 20 are marked black, while the nodes are marked gray, which makes the indexes hard to read.

\textbf{Re:}
We appreciate this advice.
Fig. 17- Fig. 20 have been modified.
We have increased the color contrast between the index and node to make the index more reader-friendly.

\vspace{1em}

% \bibliographystyle{ACM-Reference-Format}
% \bibliography{icml22}
\end{document}
